\section{Problem Statement}
In the Netherlands, the national knowledge bank on infrastructure challenges, also known as the CROW, provides traffic generation characteristics that can be used to estimate the traffic effects of spatial developments ex ante. CROW has traffic generation characteristics of motorized traffic for various purposes (living, working, shopping, etc.) by degree of urbanization (very strongly urban to non-urban) and location (centre to rural area). The focus in determining the traffic effects of spatial developments is therefore often on motorized traffic. 

There is a growing need from road authorities to understand the multimodal traffic effects. This is also increasingly reflected in the growing demand for multimodal traffic models. The STOMP principle of priorities (translated to Walking, Cycling, Public Transport, MaaS, Private car, respectively) is increasingly part of the local mobility policy in more and more municipalities. With the STOMP principle, the private car is no longer central to the design of the space, but more priority is given to sustainable forms of mobility (including walking, cycling and public transport). 

The bicycle is high on the political agenda both nationally and in local governments, but until now there is relatively little insight into the bicycle generation of various functions. In other words, the traffic effects for bicycles are (possibly) underexposed. How many cycling movements results from new developments or (re)developments? This linkage between land-use and mobility patterns is by all means a well developed study area, and has been studied in various contexts, and as of recently increasingly into active mobility.

\section{Idea \& Assignment}
The above problem definition leads to the following objective for a master-thesis research project:

\begin{quote}
    'Developing general bicycle generation characteristics by means of statistical analyses between land use characteristics and realised bike counting, with which a forecast can be made of the number of bicycle rides that a new area development will generate.'
\end{quote}
 
On the basis of one or more datasets of bicycle counting, of which there are increasingly more installed throughout the Netherlands, statistical analyzes are carried out into the bicycle use of various residential functions. This applies to both utilitarian and recreational bicycle rides. In doing so, any differences between the various residential functions as stated by CROW are also examined (for example: land-based owner-occupied home, owner-occupied apartment, private sector rental, social rental, room rental, sheltered housing, etc.). It is being investigated whether a common line can be drawn for the various functions and then translated into key figures for each residential function, divided according to degree of urbanization and location within a municipality. Possibly working towards a bicycle traffic generation for simplified CROW residential functions, e.g. detached house, ground-level house, apartment and also a number of commonly used CROW functions (such as school/supermarket/business park).